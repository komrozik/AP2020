\section{Diskussion}
\label{sec:Diskussion}
Die Messung des Röntgenspektrums von Kupfer kann den Daten zufolge als erfolgreich angesehen werden.
Wenn der Graph betrachtet wird ist klar differenzierbar wo die Peaks sind und wo der Bremsberg ist.
Für die $K_{\alpha}$ Linie hat die Energie eine relative Abweichung von $0,07\%$ vom Literaturwert und bei der $K_{\beta}$ Linie hat die Energie eine relative Abweichung von $0,1\%$.
Die Abweichung der Werte von der Literatur \cite{CuLit} ist somit sehr gering.
Bei der Transmission liegt die Ausgleichgerade passend auf den Messwerten und die Abweichung der Parameter lassen auch vermuten dass die lineare Ausgleichsgerade die Werte gut approximiert.
Die aus dem Experiment bestimmte Comptonwellenlänge weicht um $(50\pm50)\%$ vom Literaturwert \cite{Lit} ab.
Diese Abweichung kann durch eine ungenaue Berechnung der Wellenlänge aus der Transmission begründet werden.
Auch eine Streuung der Photonen im Aluminium kann dazu führen, dass nicht alle Photonen auf das Zählrohr treffen und die Comptonwellenlänge somit größer ausfällt.
Bei der Messung der Compton Wellenlänge kann die Totzeitkorrektur vernachlässigt werden, da im gemessenen Bereich nur relativ kleine Impulsraten vorkommen, die Korrektur aber nur bei hohen Raten erforderlich ist.


