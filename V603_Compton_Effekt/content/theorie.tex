\section{Ziele}
Mithilfe des Versuchs soll die Comptonwellenlänge $\lambda_{\text{C}}$ für ein Elektron bestimmt werden.

\section{Theoretische Grundlagen}
\label{sec:theorie}

\subsection{Compton Effekt}
Wenn Röntgenstrahlung mit Atomen wechselwirkt kommt es zum sogenannten Compton Effekt.
Für den Compton Effekt sind nur die Wechselwirkungen von Photonen mit Elektronen interessant.
Beim Zusammentreffen des Photons mit einem fest gebundenen Elektron auf einer inneren Schale des Atoms findet eine quasi Reflexion des Photons statt. Es gibt keinen merklichen Unterschied in der Wellenlänge.
Wenn das Photon allerdings mit einem freien Elektron in einer äußeren Schale des Atoms reagiert, findet der Compton Effekt Abb. (\ref{fig:Compton_Effekt}) statt und die Teilchen werden gestreut.
Durch die Streuung gibt das Photon Energie an das Elektron ab und seine Wellenlänge vergrößert sich (Gl. \ref{eqn:Photon_Energie}).
\begin{equation}
    E_{\text{Photon}} = \frac{hc}{\lambda} \label{eqn:Photon_Energie}
\end{equation}
Die Differenz der Wellenlänge folgt dabei folgendem Gesetz:
\begin{equation}
    \Delta \lambda = \lambda_2 - \lambda_1
                    = \frac{h}{m_e c}\left( 1- \cos \theta \right) 
                    = \lambda_{\text{C}} \left( 1- \cos \theta \right) \label{eqn:Compton_Gesetz} 
\end{equation}
Wobei die emittierte Wellenlänge bei $\theta = 0°$ minimal ($\Delta\lambda = 0$) ist und für $\theta = 180°$ maximal ($\Delta\lambda = 2\lambda_{\text{C}}$) wird.
\begin{figure}
    \centering
    \includegraphics[width=0.7\textwidth]{bilder/Compton_Effekt.png}
    \caption{Wechselwirkung eines Photons mit einem Elektron, das Photon wird im Winkel $\theta$ gestreut. (Quelle: \cite{Anleitung})}
    \label{fig:Compton_Effekt}
\end{figure}
Der Compton Effekt kann bei allen möglichen Wellenlängen stattfinden.
Im sichtbaren Bereich ist die Wellenlänge allerdings um einen Faktor $10^5$ größer als die Compton Wellenlänge, somit ist kein großer Wellenlängenunterschied spürbar.
Damit der Comptoneffekt in einem Festkörper stattfinden kann, muss die Strahlung genügend Energie haben. Somit ist sichtbares Licht zu schwach und es wird Röntgenstrahlung benötigt.
Es tritt lediglich ein anderer Effekt auf.


\subsection{Röntgenquelle}
In einer evakuierten Röhre werden mit einer Glühkathode Elektronen erzeugt und auf eine Kupferoberfläche beschleunigt.
Beim Auftreffen auf die Kupferoberfläche wird Röntgenstrahlung ausgesandt. Die Röntgenstrahlung setzt sich aus der Bremsstrahlung und charakteristischen Peaks zusammen.

\subsubsection{Bremsstrahlung}
Beim Eintritt in das Anodenmaterial treten die Elektronen auch in das Coulombfeld der Atomkerne des Kupfers ein.
Im Coulombfeld der Kerne werden die Elektronen abgelenkt und abgebremst, dadurch verringert sich ihre kinetische Energie.
Die Energiedifferenz durch die Abbremsung wird in Form von Röntgenstrahlung in einem kontinuierlichen Spektrum abgestrahlt.

\subsubsection{Charakteristische Peaks}
Wenn die Elektronen in das Anodenmaterial eindringen kann es auch vorkommen, dass die Elektronen in den Kupferatomen angeregt werden.
Wenn ein Kupfer-Elektron auf ein höheres Energieniveau angeregt wird, gibt es den Platz in der Schale frei. Auf den energetisch günstigeren Zustand "fällt" nun ein neues Elektron.
Beim "fallen" in den Zustand gibt das Elektron Energie in Form eines Röntgenquants ab.
Die Energie der Röntgenstrahlung ist somit durch die Energiedifferenz der Zustände klar definiert, dadurch enstehen charakteristische Peaks zu den entsprechenden Energien.

\subsection{Transmission}
In Aluminium wird die entstandene Röntgenstrahlung sowohl transmittiert als auch Absorbiert, wobei die Transmission Wellenlängenabhängig ist und die Absorption einem exponentiellen Absorptionsgesetz folgt(Gl. \ref{eqn:Absorption}).
\begin{equation}
    I = I_0 \cdot e^{-\mu d} \label{eqn:Absorption}
\end{equation}
Wobei $d$ die Materialdicke und $\mu = \mu_{\text{Paar}}+\mu_{\text{Photo}}+\mu_{\text{Compton}}$ der Absorptionskoeffizient ist.
Für die Transmission gilt, dass bei wachsender Wellenlänge die Transmission proportional sinkt.

\subsection{Bragg Reflexion}
Wenn Röntgenstrahlung auf ein dreidimensionales regelmäßiges Kristallgitter fällt findet die sogenannte Bragg-Reflexion statt.
Wenn die Röntgenstrahlung auf den Kristall fällt wird sie in verschieden tiefen Netzebenen des Kristalls refelektiert. Wenn ein Gangunterschied der ungleich einem Vielfachen der Wellenlänge der einfallenden Strahlung ist auftritt, dann entsteht durch die große Anzahl der Ebenen destruktive Interferenz und die Strahlungsintensität nimmt stark ab.
Falls der Entrittswinkel allerdings genau dem Bragg-Winkel (nach Gl. \ref{eqn:Bragg_Gleichung}) entspricht findet zwischen den Ebenen konstruktive Interferenz statt und die Strahlung wird mit gleicher Intensität emittiert.
\begin{equation}
    n\lambda = 2 d \sin(\alpha) = n\cdot \frac{hc}{E} \label{eqn:Bragg_Gleichung}
\end{equation}
Bragg Gleichung mit der Beugungsordnung $n$ ,dem Gitterabstand $d$ (siehe Abb. \ref{fig:Bragg_Reflexion}) und dem Glanzwinkel $\alpha$ (siehe Abb. \ref{fig:Bragg_Reflexion}).
\begin{figure}
    \centering
    \includegraphics[width=0.7\textwidth]{bilder/Bragg_Reflexion.png}
    \caption{Bragg Reflexion im Winkel $\alpha$ an einem Kristallgitter mit Gitterabstand $d$ (Quelle: \cite{Anleitung}) }
    \label{fig:Bragg_Reflexion}
\end{figure}

\subsection{Geiger Müller Zählrohr}
Die Röntgenstrahlung wird im Experiment mit einem Geiger-Müller Zählrohr gemessen.
Ein Geiger Müller Zählrohr ist ein mit Gas gefüllter Zylinder, bei dem auf einer Seite durch ein Glimmfenster die Strahlung einfallen kann.
Im Zylinder befindet sich in der Mitte ein Stab der als Anode dient und der Zylindermantel dient als Kathode.
Es wird nun der Mantel mit dem Minuspol einer Stromquelle und der Stab im Innern über einen großen Wiederstand mit dem Pluspol verbunden.
Wenn nun ein Röntgenquant in das Zahlrohr fällt ionisiert er die Atome und eine Kaskade an Ionistation durch die Elektronen und entstehende Photonen findet statt.
Zu dem Zeitpunkt, zu dem fast das ganze Rohr ionisiert ist, kann dann parallel zum Wiederstand ein Impuls gemessen werden.
Durch diese Messmethode tritt das Problem der sogenannten Totzeit auf , denn wenn ein Photon in den Zylinder fällt und dieser sich noch nicht entladen hat trägt das Photon nur zur aktuellen Ionisation bei und wird selber nicht gemessen.
Um den Effekt der Totzeit bei hoher Strahlungsintensität klein zu halten wird eine sogenannte Totzeitkorrektur (Gl. \ref{eqn:Totzeitkorrektur}) durchgeführt, bei der die Experimentelle Zählrate $N$ mit der Totzeit $\tau$ in die korrigierte Zählrate $I$ umgerechnet wird.
\begin{equation}
    I  = \frac{N}{1-\tau \cdot N}    \label{eqn:Totzeitkorrektur}
\end{equation}

