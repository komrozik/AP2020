\section{Theorie}
\label{sec:theorie}
\subsection{Ziele}
Mithilfe des Versuchs soll die Comptonwellenlänge $\lambda_{\text{C}}$ für ein Elektron bestimmt werden.
\subsection{Theorie des Versuchs}
Um die Comptonwellenlänge messen zu können muss der Compton Effekt zuerst stattfinden, 
daher wird im Versuch eine Plexiglasscheibe aufgebaut in der Röntgenstrahlung mit Elektronen wechselwirken kann.
Bei der kohärenten Streuung im Plexiglas wechselwirken die Photonen durch den Compton Effekt mit den Elektronen
und es werden Elektronen und die Rontgenstrahlung mit verschobener Wellenlänge emittiert.
Für die Wechselwirkung nach Abb. (\ref{Abb:Compton}) gilt:
\begin{equation}
    \Lambda \lambda = \lambda_2 - \lambda_1
                    = \frac{h}{m_e c}\left( 1- \cos \theta \right) 
                    = \lambda_{\text{C}} \left( 1- \cos \theta \right) \label{eqn:wellenlänge} 
\end{equation}
Wobei die emittierte Wellenlänge bei $\theta = 0°$ minimal ($\lambda = 0$) ist und für $\theta = 180°$ maximal ($\lambda = 2\lambda_{\text{C}}$) ist.
Um den Compton Effekt zu realisieren muss die benötigte Röntgenstrahlung zuerst noch erzeugt werden.
In einer evakuierten Röhre werden daher mit einer Glühkathode elektronen erzeugt und auf eine Kupfer Oberfläche beschleunigt.
In der Kupferoberfläche werden die Elektronen im Coulombfeld der Atomkerne abgelenkt und ein kontinuierliches Spektrum von Bremsstrahlung wird emittiert.
Außerdem werden die Elektronen der Kupferatome angeregt und wechseln nach der Anregung direkt wieder in einen energetisch günstigeren Zustand, wobei sie die chrakteristische Strahlung des Kupfers emittieren.

Transmission 
In ALuminium wird die entstandene Röntgenstrahlung transmittiert als auch Absorbiert, wobei die Transmission Wellenlängenabhängig ist und die Absorption einem exponentiellen Absorptionsgesetz folgt(\ref{eqn:Absorption}).
\begin{equation}
    I = I_0 \cdot e^{-\mu d} \label{eqn:Absorption}
\end{equation}
Wobei $d$ die Materialdicke und $\mu = \mu_{\text{Paar}}+\mu_{\text{Photo}}+\mu_{\text{Compton}}$ der Absorptionskoeffizient ist.
Für die Transmission gilt, dass bei wachsender Wellenlänge die Transmission sinkt.
Um die Energie und die Wellenlänge der Röntgenstrahlung zu bestimmen wird auch die Bragg reflexion genutzt.


