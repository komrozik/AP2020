\section{Diskussion}
\label{sec:Diskussion}
Aus der Messung der Charakteristik des Zählrohrs wurde die Steigung des Plateaus als $m_\% = (1,26 \pm 0,26)\%$ pro 100 V bestimmt.
Da eine geringe Steigung positiv zu bewerten ist, arbeitet das Zählrohr annähernd optimal.
Allerdings ist anzumerken,dass bei der Messung die Impulsrate über 100 Impulse pro Sekunde steigt. Impulsraten die über diesem Wert liegen sollten eigentlich vermieden werden.
Für die Messung der Totzeit ist der Wert aus der Zwei-Proben-Methode sehr nah am aus dem Oszilloskop abgelesenen Wert dran.
Der Wert aus der Zwei-Proben-Methode ist aber als zuverlässiger zu bewerten, da beim ablesen des Oszilloskops leicht Fehler entstehen.
Die im letzten Teil des Versuches berechnete Ladungsmenge pro teilchen ist wie in der Abbildung erkennbar proportional zur Stromstärke, wie es auch aus der Theorie zu vermuten ist.
Mit dem Versuch konnten die Eigenschaften des Geiger-Müller-Zählrohrs gut und zuverlässig bestimmt werden.