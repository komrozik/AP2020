\section{Auswertung}
\label{sec:Auswertung}
Die aufgenommenen Daten für die charakteristik sind Poissonverteilt und werden daher mit dem entsprechenden Fehler versehen.
Um das Plateu auszuwerten werden die Start und End Werte abgelesen und eine lineare Regression gemäß Formel \ref{eqn:linReg} gemacht.
\begin{figure}
    \centering
    \includegraphics[width=0.5\textwidth]{plots/Kennlinie.pdf}
    \caption{Eine Kennlinie.}
    \label{fig:Kennlinie}
\end{figure}
\begin{align}
    f(x) = m\cdot x +b \label{eqn:linReg}\\
    m = 1,21 \pm 0,26&& b = (9,56 \pm 0,13)\cdot 10^{3}
\end{align}
\begin{figure}
    \centering
    \includegraphics[width=0.5\textwidth]{plots/Ladungsmenge.pdf}
    \caption{Eine Ladungsmenge.}
    \label{fig:Ladungsmenge}
\end{figure}
Nun kann die Steigung der Geraden in \% pro  100V angegeben werden.