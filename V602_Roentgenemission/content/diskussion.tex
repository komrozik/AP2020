\section{Diskussion}
\label{sec:Diskussion}
Die Messung des Bragg Winkels hat eine Abweichung von 0,2° $\hat{=} 0,71\%$ ergeben.
Die Abweichung kann durch Unregelmäßigkeiten des LiF Kristalls erklärt werden.
Auch mit der Abweichung liegt der Wert noch nah genug am erwarteten Maximum um die Theorie zu bestätigen.
Wenn das Intensitätsmaximum um einen größeren Winkel verschoben wäre, würde damit auch das Spektrum für die Kupfer Emission verschoben werden und man würde fehlerhafte Energien erhalten.
Wenn beispielsweise bei der Überprüfung der Bragg Bedingung das Geiger Müller Zählrohr um 3° verschoben ist, kann es auch vorkommen, dass im nachhinein beim Spektrum zu geringe Intensitäten gemessen werden, da immer leicht neben dem eigentlichen Maximum gemessen wird.
Die Auflösungsvermögen der Peaks sind im Experiment genau genug bestimmt, somit muss der statistische Fehler nicht betrachtet werden.
Die Werte für $\sigma_2$ und $\sigma_3$ wurden direkt aus den Energien der Peaks berechnet und ihre Abweichung gibt somit auch Rückschlüsse auf die Genauigkeit der Messung der Energiepeaks.
Die Sigmas haben beide eine kleine relative Abweichung von weniger als 3\% und können somit als repräsentativ für den Literaturwert angesehen werden.
Durch die geringe Abweichung der Absorptionskoeffizienten ist anzunehmen, dass die entsprechenden Energien der Peaks auch ausreichend genau bestimmt werden.
Die Abweichung ist durch die zuvor besprochene Ungenauigkeit des Bragg Winkels auch verstärkt.
Die maximale beziehungsweise minimale Energie des Bremsbergs lässt sich anhand des Plots nicht bestimmen, da der entsprechende Winkelbereich nicht aufgenommen wird.
Im weiteren Verlauf des Versuchs wurden noch die Abbsorptionskoeffizienten verschiedener Materialien berechnet.
Die relative Abweichung der Koeffizienten vom zuvor berechneten Literaturwert liegt in allen Fällen unter 5\%, daher können mit den gemessenen Werten die Theoriewerte bestätigt werden.
Für die Messung der Rydberg Energie aus dem Moseley Gesetz ergibt sich eine relative Abweichung von 7,5\%. 
Die Abweichung ist nicht optimal, reicht aber aus um den Literaturwert zu bestätigen.
Die Abweichung könnte dadurch zu erklären sein, dass das Moseley Gesetz eigentlich von der effektiven Ordnungszahl abhängt, bei der Näherung allerdings nur die Ordnungszahl verwendet wurde.


\begin{table}[H]
\centering
\begin{tabular}{lllll}
 Messgröße & Messwert & Literaturwert & Relative \\
  & & & Abweichung\\
  \toprule
Bragg-Winkel & 28,2° & 28° &  0,71\%  \\
\midrule
Auflösungsvermögen $\alpha$ & 48,56 & - & - \\
\midrule
Auflösungsvermögen $\beta$ & 43,36 & - & - \\
\midrule
$\sigma_2$ für Kupfer & 12,41 & 12,36 & 0,38\% \\
\midrule
$\sigma_3$ für Kupfer & 22,40 & 21,96 & 2,03\%\\
\midrule
$\sigma_{\text{K}}$ für Zink & 3,64  & 3,57 & 1,93\%\\
\midrule
$\sigma_{\text{K}}$ für Gallium & 3,70 & 3,62 & 2,32\%\\ 
\midrule
$\sigma_{\text{K}}$ für Brom & 3,84 & 3,85 & 0,29\%\\ 
\midrule
$\sigma_{\text{K}}$ für Rubidium & 4,12 & 3,95 & 4,18\%\\ 
\midrule
$\sigma_{\text{K}}$ für Strontium & 4,13 & 4,01 & 4,07\%\\
\midrule
$\sigma_{\text{K}}$ für Zirkonium & 4,29 & 4,11 & 4,40\%\\
\midrule
Rydberg-Energie & $(2,016\pm 0,025)\cdot 10^{-18} \text{Joule}$ & $2,179 \cdot 10^{-18} \text{Joule}$ & 7,50\% \\
\bottomrule
\end{tabular}
\end{table}