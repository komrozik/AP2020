\section{Auswertung}
\label{sec:Auswertung}
\subsection{Vorbereitung}
Um das Experiment vorzubereiten sollen die Energie für die $K_{\alpha}$ und $K_{\beta}$ als Literaturwerte recherchiert werden.
Außerdem soll zu den Energien der entsprechende Winkel berechnet werden (Gleichung \ref{eqn:Bragg_Bedingung}).
\begin{align}
    E_{K_{\alpha}} &= 8,038 \text{keV} && E_{K_{\beta}} = 8,905 \text{keV} \nonumber \\
    \theta_{\alpha} &= 22,52°    &&      \theta_{\beta} = 20,22° \nonumber
\end{align}
Zu den Elementen bei denen die Absorption untersucht wird sollen auch die entsprechenden Werte recherchiert werden und die Parameter $\theta_K^{Lit}$ und $\sigma_K$ bestimmt werden.
\begin{table}[H]
\centering
\begin{tabular}{lllll}
  & Z & $E_K^{Lit}$ & $\theta_K^{Lit}$ & $\sigma_K$\\
  &   & [keV] & [°] &  \\
  \toprule
Zn & 30 & 9,65 & 18,6 & 3,56\\
\midrule
Ga & 31 & 10,37 & 17,27 & 3,62\\
\midrule
Ge & 32 & 11,10 & 16,1 & 3,68\\ 
\midrule
Br & 35 & 13,47 & 13,2 & 3,85\\ 
\midrule
Rb & 37 & 15,20 & 11,7 & 3,95\\ 
\midrule
Sr & 38 & 16,10 & 11,0 & 4,01\\
\midrule
Zr & 40 & 17,99 & 9,6 & 4,11\\
\bottomrule
\end{tabular}
\end{table}
Die Parameter wurden aus der Energie mit der Bragg Bedingung \ref{eqn:Bragg_Bedingung} und der Formel \ref{eqn:Sigma} berechnet.
Die Werte für Gerundium sind für unseren Versuch uninteressant, da das Material nicht untersucht wird.
\subsection{Bragg Bedingung}
Aus den Messdaten ergibt sich der Plot \ref{fig:Bragg} und es kann mit der SciPy Funktion findPeaks das Maximum bestimmen.
Damit lässt sich auch die Abweichung von den Literaturwerten bestimmen.
\begin{align}
    \Delta \theta_{\text{abs}} = \theta_{\text{lit}}-\theta_{\text{exp}} && \Delta \theta_{\text{rel}} = \frac{\Delta \theta_{\text{abs}}}{\theta_{\text{lit}}}\\
    \Delta \theta_{\text{abs}} = 0,2° &&     \Delta \theta_{\text{rel}} = 0,0071 \; \hat{=} \; 0,71 \% 
\end{align}
\begin{figure}
    \centering
    \includegraphics[width=0.7\textwidth]{plots/Bragg.pdf}
    \caption{Ein Plot}
    \label{fig:Bragg}
\end{figure}

\subsection{Kupfer Emission}
Aus den Messdaten lässt sich der Plot \ref{fig:Cu_Emission} erstellen und die Peaks und der Bremsberg sind klar zu erkennen.
\begin{figure}
    \centering
    \includegraphics[width=0.7\textwidth]{plots/Cu_Emission.pdf}
    \caption{Ein Kupfer Plot}
    \label{fig:Cu_Emission}
\end{figure}
Die Peaks des Spektrums werden mit einer Scipy funktion bestimmt und der Abschnitt der Peaks nochmal vergrößert dargestellt.
\begin{figure}
    \centering
    \includegraphics[width=0.7\textwidth]{plots/Cu_Peaks.pdf}
    \caption{Ein Kupfer Plot mit tolleren Peaks}
    \label{fig:Cu_Peaks}
\end{figure}
Über eine weitere Funktion wird nun die "Full Width at Half Maximum" bestimmt.
Aus den Werten der FWHM kann nun das Auflösungsvermögen $A$ bestimmt werden.
\begin{align}
    A = \frac{E_{\text{peak}}}{\Delta E_{\text{FWHM}}} \nonumber\\
    A_{\alpha}= 48,56 && A_{\beta} = 43,36
\end{align}
Desweiteren werden die Abschirmkonstanten für Kupfer bestimmt.
Aus der Literatur findet man den Wert für $E_{\text{K,abs}}$ als $8980,22$ eV.
Damit lassen sich die Abschirmkonstanten $\sigma$ bestimmen.
\begin{align*}
    \sigma_1 &= Z-\sqrt{\frac{E_{\text{K,abs}}}{R_{\text{y}}}} \\
    \sigma_2 &= Z-\sqrt{ \frac{m^2}{n^2}\left(z-\sigma_1\right)^2- \frac{E_{\text{K,}\alpha}}{R_{\text{y}}}m^2 } \\
    \sigma_3 &= Z-\sqrt{ \frac{l^2}{n^2}\left(z-\sigma_1\right)^2- \frac{E_{\text{K,}\beta}}{R_{\text{y}}}l^2 }
\end{align*}
\begin{align}
    \sigma_1 = 3,31 && \sigma_2 = 12,41 && \sigma_3 = 22,40 
\end{align}

\subsection{Andere Metalle}
Im letzten Teil des Experiments werden die Abschirmkonstanten der K-Schale für verschiedene Materialien bestimmt.
Aus den Messdaten lässt sich für die Metalle jeweils ein Plot darstellen und die Lage der K Kante anhand von Maximum und Minimum berechnen (\ref{eqn:Kante}).
Mit der berechnetetn Lage der K-Kante kann nun der Wert von $\theta$ für die jeweilige K-Kante abgelesen werden. 
\begin{equation}
    I_{\text{K}}= I_{\text{K}}^{\text{min}} + \frac{I_{\text{K}}^{\text{max}}-I_{\text{K}}^{\text{min}}}{2} \label{eqn:Kante}
\end{equation}
Aus den Werten von $\theta$ lässt sich mit Gleichung \ref{eqn:Bragg_Bedingung} die entsprechende Energie der K-Kante berechnen.
Aus der Energie kann mit Gleichung \ref{eqn:Sigma} der Absorptionskoeffizient berechnet werden.
\begin{table}[H]
\centering
\begin{tabular}{lllll}
  & Z & $I_{\text{K}}$ & $E_{\text{K}}$ & $\sigma_K$\\
  &   & [Imp/s]& [eV] &   \\
  \toprule
Zn & 30 & 78,5 & 9601,30 & 3,64 \\
\midrule
Ga & 31 & 93,5 & 10308,27 & 3,70 \\ 
\midrule
Br & 35 & 18,0 & 13480,57 & 3,84\\ 
\midrule
Rb & 37 & 38,0 & 15053,11 & 4,12 \\ 
\midrule
Sr & 38 & 121,5 & 15989,35 & 4,13 \\
\midrule
Zr & 40 & 197,0 & 17815,40 & 4,29 \\
\bottomrule
\end{tabular}
\end{table}

\subsection{Moseley Gesetz}
Laut dem Moseley Gesetz (\ref{eqn:Moseley}) ist die Absorptionsenergie der Kanten proportional zu $z^2$ und aus einer entsprechenden linearen Ausgleichsrechnung lässt sich $R$ bestimmen.
\begin{equation}
    E_{\text{k}}= Rh\left(z-\sigma\right)^2 \label{eqn:Moseley}
\end{equation}
\begin{figure}
    \centering
    \includegraphics[width=0.7\textwidth]{plots/Rydberg.pdf}
    \caption{Ein Moseley Plot}
    \label{fig:Moseley}
\end{figure}