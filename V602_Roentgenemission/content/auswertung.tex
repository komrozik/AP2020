\section{Auswertung}
\label{sec:Auswertung}
\subsection{Vorbereitung}
Um das Experiment vorzubereiten sollen die Energie für die $K_{\alpha}$ und $K_{\beta}$ als Literaturwerte recherchiert werden.
Außerdem soll zu den Energien der entsprechende Winkel berechnet werden (Gleichung \ref{eqn:Bragg_Bedingung}).
\begin{align}
    E_{K_{\alpha}} = 8,038 \text{keV} && E_{K_{\beta}} = 8,905 \text{keV} \\
    \theta_{\alpha} = 22,52°    &&      \theta_{\beta} = 20,22°
\end{align}
Zu den Elementen bei denen die Absorption untersucht wird sollen auch die entsprechenden Werte recherchiert werden und die Parameter $\theta_K^{Lit}$ und $\sigma_K$ bestimmt werden.
\begin{table}
\centering
\begin{tabular}{lllll}
  & Z & $E_K^{Lit}$ & $\theta_K^{Lit}$ & $\sigma_K$\\
  \toprule
Zn & 30 & 9,65 & 18,6 & 3,56\\
\midrule
Ge & 32 & 11,10 & 16,1 & 3,68\\ 
\midrule
Br & 35 & 13,47 & 13,2 & 3,85\\ 
\midrule
Rb & 37 & 15,20 & 11,7 & 3,95\\ 
\midrule
Sr & 38 & 16,10 & 11,0 & 4,01\\
\midrule
Zr & 40 & 17,99 & 9,6 & 4,11\\
\bottomrule
\end{tabular}
\end{table}

\subsection{Bragg Bedingung}
\begin{figure}
    \centering
    \includegraphics[width=0.7\textwidth]{plots/Bragg.pdf}
    \caption{Ein Plot}
    \label{fig:Bragg}
\end{figure}

\subsection{Kupfer Emission}
\begin{figure}
    \centering
    \includegraphics[width=0.7\textwidth]{plots/Cu_Emission.pdf}
    \caption{Ein Kupfer Plot}
    \label{fig:Cu_Emission}
\end{figure}

\subsection{Andere Metalle}
\begin{figure}
    \centering
    \includegraphics[width=0.7\textwidth]{plots/Brom.pdf}
    \caption{Ein Brom Plot}
    \label{fig:Brom}
\end{figure}
\begin{figure}
    \centering
    \includegraphics[width=0.7\textwidth]{plots/Gallium.pdf}
    \caption{Ein Gallium Plot}
    \label{fig:Gallium}
\end{figure}
\begin{figure}
    \centering
    \includegraphics[width=0.7\textwidth]{plots/Rubidium.pdf}
    \caption{Ein Rubidium Plot}
    \label{fig:Rubidium}
\end{figure}
\begin{figure}
    \centering
    \includegraphics[width=0.7\textwidth]{plots/Strontium.pdf}
    \caption{Ein Strontium Plot}
    \label{fig:Strontium}
\end{figure}
\begin{figure}
    \centering
    \includegraphics[width=0.7\textwidth]{plots/Zink.pdf}
    \caption{Ein Zink Plot}
    \label{fig:Zink}
\end{figure}
\begin{figure}
    \centering
    \includegraphics[width=0.7\textwidth]{plots/Zirkonium.pdf}
    \caption{Ein Zirkonium Plot}
    \label{fig:Zirkonium}
\end{figure}